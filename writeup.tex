\documentclass[twoside]{article}
\usepackage{paper}

\title{\vspace{-15mm}\fontsize{24pt}{10pt}\selectfont\textbf{\titlee}} % Article title

\author{
\large
\textsc{Manish Raghavan, Serena Gupta}
\vspace{-5mm}
}

\begin{document}

\maketitle % Insert title

\thispagestyle{fancy} % All pages have headers and footers

\begin{abstract}

  Research shows that people aren't rational in evaluating costs -- instead,
  they undervalue costs in a nonlinear fashion
  \cite{Kahneman1979}. We consider the effects of this bias on agents in atomic congestion
  games, bounding its negative effects on social performance. We show regardless
  of the class of delay functions used, the ratio between the performance of
  biased agents and the social optimum grows with the size of the game. In other words, our findings indicate that people misunderstand the additivity of
  costs, preferring one large cost over a large number of smaller costs, even
  when it is not in their rational interest to do so.

$\xi$

\end{abstract}

\begin{multicols}{2} % Two-column layout throughout the main article text

\section{Introduction}
Prospect theory, introduced by Kahneman and Tversky in 1979, characterizes some
of the biases humans exhibit in decision-making \cite{Kahneman1979}. Roughly
speaking, it proposes the idea that humans tend to overweight small
probabilities and undervalue large rewards and costs. Experimental evidence
shows people are \textit{risk-seeking} when it comes to costs -- for example,
people prefer an 80\% chance of losing \$4,000 to a sure loss of \$3,000,
despite the expected loss being higher \cite{Kahneman1979}.

As a result, classical game theory fails to accurately characterize what happens
when people interact with each other. Accounting for these biases in games has
lead to ``behavioral game theory,'' which assumes that biased agents perceive a
different version of the game, within which they act rationally \cite{Meir2014}.
This paper seeks to bring prospect theory and behavioral game theory together,
the combination of which, to our knowledge, is novel.

\section{A Motivating Example}
Our work is motivated by real life road congestion.  In familiar networks, on a clear day you know what the congestion will be; call this a reference point.  And when you wake up late or it's raining, there ends up being more congestion on the road than you think.  To be specific, imagine you can take two roads to get to work: one that usually takes 20 minutes and with traffic takes an additional 9 and the other road takes 24 minutes and with traffic an additional 4 minute.  Now you probably are still biased to go on the road you know is usually less trafficy, the 20 + 9 = 29 minute road, instead of the 24 + 4 = 28 minute road.  If we assume people undervalue additional costs by taking the square root of the additional cost, we find that this exactly models our intuitive understanding as the perceived cost is now 20 + 3 = 23 for one road and 24 + 2 = 26 for the other road so you want to take the road with a base cost of 20.

\section{Related Work}
The price of anarchy (POA) is defined as the worst-case ratio between a Nash
equilibrium and the socially optimal solution in a game, and intuitively gives
an upper bound on how closely the Nash equilibrium approximates the optimal
solution \cite{Koutsoupias2009}. It has been extensively studied in atomic
congestion games with rational agents, with tight matching lower and upper
bounds shown for various classes of cost funtions
\cite{Aland2011,Roughgarden2012}. Tim Roughgarden introduced the smooth analysis
framework to not only bound the price of anarchy, but also to construct
worst-case examples to provide matching lower bounds \cite{Roughgarden2012}.

While classical game theory relies on the assumption of rational agents, recent
work has sought to model agents with various behavioral biases. For example,
Reshef Meir and David Parkes consider the effects of behavioral biases in
nonatomic congestion games, applying smoothness arguments to upper bound the
``biased price of anarchy'' \cite{Meir2014}. In particular, they show that the
biased price of anarchy can be bounded as long as the agents being considered
satisfy a certain smoothness property with respect to the class of delay
functions; however, this approach is not applicable for the agents we consider
because the biased price of anarchy is unbounded for any class of delay
functions.

\section{Overview of Results}
Broadly, we answer the question ``what is the price of anarchy when agents
underestimate costs?'' We call \textit{lowball agents}, agents who underestimate
costs following the behavioral models from prospect theory, and study the ratio between the true cost of the Nash equilibrium reached by lowball agents and the overall optimum solution in atomic congestion games.

In Section~\ref{sec:lb}, we show that for an atomic congestion game with
socially optimal strategy profile $\s^*$, the \textit{lowball price of anarchy} is
$\Omega(C(\s^*)^{1/\beta-1})$. In Sections~\ref{sec:ub} and~\ref{sec:ubl}, we
show a matching upper bound of $O(C(\s^*)^{1/\beta - 1})$, as long as the delay
functions being used are ``smooth.''

\section{The Model}

Here we present a framework to model lowball agents in congestion games. We define a 
\textit{lowball agent} to evaluate the added cost to some baseline value according to
\begin{equation}
  V(x) = x^{\beta}
  \label{eq:val}
\end{equation}
defined by Rieger and Wang for $0 < \beta \le 1$ \cite{Rieger2008}. For the
remainder of this paper, we assume that each agent has the same parameter
$\beta$.

An atomic congestion game, $\mathcal{G} = (G, D, P, k)$, is a cost-minimization
game parametrized by a directed graph $G$, a set of nondecreasing delay
functions $D$ such that each function is defined by $c_e(x) = \tau_e + \Delta_e(x)$ where $\Delta_e$ is a nondecreasing added cost function, a set of start and end
points for each player $P =\{(s_1, t_1), \cdots, (s_k, t_k)\}$, and $k$ players.

As we noted before, lowball agents perceive congestion games by applying the valuation
function~\eqref{eq:val} to each edge's added cost function, $\Delta_e(x)$, leading them to
perceive the game as $\mathcal{G}\pbet= (G, D\pbet, P, k)$ where $D\pbet =
\{c_e\pbet(x)\}$, where $c_e\pbet(x) = \tau_e + \Delta_e\pbet(x) = \tau_e + (\Delta_e(x))^\beta$. Within this new
congestion game $\mathcal{G}\pbet$, the lowball agents act rationally: however
the true costs they incur are reflected by the delay functions in $\mathcal{G}$.   

Let $S_i$ be the set of all strategies for player $i$ in $\mathcal{G}$ i.e. all
paths that go from $s_i$ to $t_i$.  Note $S_i$ is also the set of all strategies for
player $i$ in $\mathcal{G}\pbet$.  Since the true costs agents incur is
reflected in $\mathcal{G}$, it's useful to look at the difference in cost of an
agent $i$'s strategy $s_i$ with the
cost functions in $\mathcal{G}\pbet$ versus with the cost functions in $\mathcal{G}$.
Rational play in $\mathcal{G}\pbet$ amounts to the players playing according to
a Nash equilibrium.  Furthermore we know from Rosenthal \cite{Rosenthal1973}
that every atomic congestion game has a pure Nash equilibrium.  Thus for a given
pure Nash equilibrium strategy profile $\mathbf{s} = (s_1, \cdots, s_k)$ with
each $s_i \in S_i$ for each player $i$, we can evaulate the perceived cost of
$\mathbf{s}$ (the cost in $\mathcal{G}\pbet$) and actual cost that will be
incurred (the cost in $\mathcal{G}$).

In addition, we define the true cost to player $i$ for a strategy profile
$\mathbf{s}$ to be
\[
  C_i(\s) = \sum\limits_{e \in s_i}c_e(x_e),
\]
where $\mathbf{x}
\in \Z^E$ is the vector of the number of players on each edge induced by
$\mathbf{s}$.  The perceived cost to player $i$ is
\[
  C\pbet_i(\s) = \sum\limits_{e \in s_i}c_e\pbet(x_e) = \sum\limits_{e \in
  s_i}\p{\tau_e + \Delta_e(x_e)^\beta}.
\]
From this, we know the total social cost is
\[
  C(\mathbf{s}) = \sum_{i=1}^{k}C_i(\mathbf{s}).
\]
        
Traditionally, the \textit{price of anarchy (POA)} is defined as the ratio
between the worst Nash equilibrium and the overall optimum solution
\cite{Koutsoupias2009}.  In more precise terms, the price of anarchy for a
family of games $\hat{\mathcal{G}}$ is
\[
  \sup_{\mathcal{G} \in \hat{\mathcal{G}}} \frac{C(\s)}{C(\s^*)},
\]
where $\s$ is a Nash equilibrium of $\mathcal{G}$ and $\s^*$ is the overall
optimum strategy profile of $\mathcal{G}$.

\begin{defn}
  The \textit{lowball price of anarchy (LPOA)} for a family of congestion games
  $\hat{\mathcal{G}}$ is
  \begin{equation}
    \sup_{\mathcal{G} \in \hat{\mathcal{G}}} \frac{C(\s\pbet)}{C(\s^*)},
    \label{eq:lpoa}
  \end{equation}
  where $\s\pbet$ is a Nash equilibrium of $\mathcal{G}\pbet$ and $\s^*$ is the
  socially optimum strategy profile of $\mathcal{G}$.
\end{defn} In other words, the LPOA is a ratio between the true cost of the Nash
equilibrium reached by lowball agents and the overall optimum solution.

\begin{figure}[H]
  \centering
  \begin{subfigure}[b]{\linewidth}
    \centering
    \begin{tikzpicture} [baseline=(s.base), arc/.style={->,thick,>=stealth},
      vertex/.style={draw,circle,minimum size=.5cm},scale=1.5]

      \foreach \l/\n/\p in {{s/s/(0,0)}, {t/t/(2,0)}}
      {
        \node [vertex] at \p (\l) {$\n$};
      }

      \draw [arc,bend left=30] (s) to node [auto] {$20 + \textcolor{red}{9}$} (t);
      \draw [arc,bend right=30] (s) to node [auto] {$24 + \textcolor{red}{4}$}
      (t);
    \end{tikzpicture}
    \caption{Road congestion}
    \label{fig:road}
  \end{subfigure}

  \begin{subfigure}[b]{\linewidth}
    \centering
    \begin{tikzpicture} [baseline=(s.base), arc/.style={->,thick,>=stealth},
      vertex/.style={draw,circle,minimum size=.5cm},scale=1.5]

      \foreach \l/\n/\p in {{s/s/(0,0)}, {t/t/(2,0)}}
      {
        \node [vertex] at \p (\l) {$\n$};
      }

      \draw [arc,bend left=30] (s) to node [auto] {$20 + \textcolor{red}{3}$} (t);
      \draw [arc,bend right=30] (s) to node [auto] {$24 + \textcolor{red}{2}$}
      (t);
    \end{tikzpicture}
    \caption{Perceived road congestion}
    \label{fig:roadbet}
  \end{subfigure}
  \caption{Actual vs. perceived road congestion}
\end{figure}

\section{Example with Affine Delay Functions}
\begin{figure}[H]
  \centering
  \begin{subfigure}[b]{\linewidth}
    \centering
    \begin{tikzpicture} [baseline=(s.base), arc/.style={->,thick,>=stealth},
      vertex/.style={draw,circle,minimum size=.5cm},scale=1.5]

      \foreach \l/\n/\p in {{s/s/(0,0)}, {v1/ /(1,0)}, {v2/ /(2,0)},
      {t/t/(3,0)}}
      {
        \node [vertex] at \p (\l) {$\n$};
      }

      \foreach \a/\b in {{s/v1}, {v1/v2}, {v2/t}}
      {
        \Arc {\a}{\b}{$x$}
      }

      \draw [arc,bend left=40] (s) to node [auto] {$2 \cdot 3^{1/\beta}$} (t);
    \end{tikzpicture}
    \caption{A game $\mathcal{G}$ with affine delay functions}
    \label{fig:affine}
  \end{subfigure}

  \begin{subfigure}[b]{\linewidth}
    \centering
    \begin{tikzpicture} [baseline=(s.base), arc/.style={->,thick,>=stealth},
      vertex/.style={draw,circle,minimum size=.5cm},scale=1.5]

      \foreach \l/\n/\p in {{s/s/(0,0)}, {v1/ /(1,0)}, {v2/ /(2,0)},
      {t/t/(3,0)}}
      {
        \node [vertex] at \p (\l) {$\n$};
      }

      \foreach \a/\b in {{s/v1}, {v1/v2}, {v2/t}}
      {
        \Arc {\a}{\b}{$x^\beta$}
      }

      \draw [arc,bend left=40] (s) to node [auto] {$2^\beta\cdot3$} (t);
    \end{tikzpicture}
    \caption{The game $\mathcal{G}\pbet$ that the agents perceive}
    \label{fig:affinebet}
  \end{subfigure}
  \caption{The original game $\mathcal{G}$ and the perceived game
  $\mathcal{G}\pbet$}
\end{figure}
Consider the two-player game shown in Figure~\ref{fig:affine} where for all edges $e$, $\tau_e = 0$. Assume $\beta =
1/2$. Then, the cost of taking the top path is $2 \cdot 9 = 18$. However, if
both players take the bottom path, each incurs a cost of 6. Thus, for rational
agents, the Nash equilibrium and the socially optimal strategy are the same --
both players take the bottom path.

However, through the valuation function~\eqref{eq:val}, the agents perceive this
as the game shown in Figure~\ref{fig:affinebet}. In $\mathcal{G}\pbet$, if one
player takes the top path, the second player is indifferent between the top and
bottom paths, as they both have a perceived cost of $3 \cdot 2^\beta$. Thus, a
valid Nash equilibrium is for one player to take the bottom path and the other
to take the top path, even though this is not a Nash equilibrium in the original
game. The true social cost of this equilibrium is $2 \cdot 3^{1/\beta} + 3 =
21$, while the social cost of the optimal strategy profile is 12. Clearly, this
bias can induce lowball agents to perform worse than rational agents in some
cases. We seek to understand how much worse they can do.
% \section{Preliminaries}
% \begin{lem} \label{lem:scale}
%   Scaling the cost functions by a constant $a > 0$, i.e. $\hat c_e(x) = a
%   c_e(x)$, does not change the ratio $C(\s\pbet)/C(\s^*)$.
% \end{lem}
% \begin{proof}
%   $\s\pbet$ is a Nash equilibrium in $\mathcal{G}\pbet$ if and only if $\sum_{e
%   \in s_i} c_e\pbet(x_e) \le \sum_{e \in s_i'} c_e\pbet(x_e + 1)$ for all $i$
%   and for any $s_i'$. After scaling the cost functions by $a$, $\s\pbet$ is
%   still a Nash equilibrium because for all $i$,
%   \begin{align*}
%     \sum_{e \in s_i} \hat c_e\pbet(x_e) &= a^\beta \sum_{e \in s_i}
%     c_e\pbet(x_e) \\
%     &\le a^\beta \sum_{e \in s_i'} c_e\pbet(x_e + 1) \\
%     &= \sum_{e \in s_i'} \hat c_e\pbet(x_e+1).
%   \end{align*}
%   Since the strategy profile remains the same, the ratio between costs is $a
%   C(\s\pbet)/(a C(\s^*)) = C(\s\pbet)/C(\s^*)$.
% \end{proof}
% % Because $V(x) = x^\beta$ is concave, $V$ is subadditive, i.e.
% % \begin{equation}
% %   (x+y)^\beta \le x^\beta + y^\beta
% %   \label{eq:subadditive}
% % \end{equation}
% % for $x,y \ge 0$.
\section{Lower Bound on LPOA} \label{sec:lb}
\begin{figure}[H]
  \centering
  \begin{subfigure}[b]{\linewidth}
    \centering
    \begin{tikzpicture} [baseline=(s.base), arc/.style={->,thick,>=stealth},
      vertex/.style={draw,circle,minimum size=.5cm},scale=1.5]

      \foreach \l/\n/\p in {{s/s/(0,0)}, {v1/ /(1,0)}, {v2/ /(3,0)},
      {t/t/(4,0)}}
      {
        \node [vertex] at \p (\l) {$\n$};
      }
      \node at (2,0) (d) {$\dots$};
      \draw
      [-,thick,decorate,decoration={brace,amplitude=10pt}](3,-0.2) -- (0,-0.2)
      node[black,midway,yshift=-0.5cm]
      {$n$};

      \foreach \a/\b in {{s/v1}, {v1/d}, {d/v2}, {v2/t}}
      {
        \Arc {\a}{\b}{1}
      }

      \draw [arc,bend left=40] (s) to node [auto] {$n^{1/\beta}$} (t);
    \end{tikzpicture}
    \caption{A game $\mathcal{G}$ with non-constant LPOA}
    \label{fig:lower}
  \end{subfigure}

  \begin{subfigure}[b]{\linewidth}
    \centering
    \begin{tikzpicture} [baseline=(s.base), arc/.style={->,thick,>=stealth},
      vertex/.style={draw,circle,minimum size=.5cm},scale=1.5]

      \foreach \l/\n/\p in {{s/s/(0,0)}, {v1/ /(1,0)}, {v2/ /(3,0)},
      {t/t/(4,0)}}
      {
        \node [vertex] at \p (\l) {$\n$};
      }
      \node at (2,0) (d) {$\dots$};
      \draw
      [-,thick,decorate,decoration={brace,amplitude=10pt}](3,-0.2) -- (0,-0.2)
      node[black,midway,yshift=-0.5cm]
      {$n$};

      \foreach \a/\b in {{s/v1}, {v1/d}, {d/v2}, {v2/t}}
      {
        \Arc {\a}{\b}{1}
      }

      \draw [arc,bend left=40] (s) to node [auto] {$n$} (t);
    \end{tikzpicture}
    \caption{The game $\mathcal{G}\pbet$ that the agent perceives}
    \label{fig:lowerbet}
  \end{subfigure}
  \caption{A lower bound on the LPOA}
\end{figure}

As a lower bound, consider the LPOA in the one-player game shown in
Figure~\ref{fig:lower} where for all edge $e$, $\tau_e = 0$. Through its valuation function~\eqref{eq:val}, the agent
perceives the game as in Figure~\ref{fig:lowerbet}. Thus, it is indifferent
between the upper and lower paths, so taking the upper path is a valid
equilibrium. However, the overall cost of this strategy is $n^{1/\beta}$,
whereas the optimal social cost is $C(\s^*) = n$, corresponding to the agent
taking the lower path. Thus, the performance ratio is $n^{1/\beta}/n =
n^{1/\beta-1} = C(\s^*)^{1/\beta-1}$.

Considering the cost functions $c_e$ to be constants, this can be extended to
arbitrarily many players: if $k$ players play this game, a valid Nash
equilibrium $\s\pbet$ corresponds to all players taking the top path, incurring
a total cost of $C(\s\pbet) = kn^{1/\beta}$, when the optimal strategy profile
$\s^*$ corresponds to all agents taking the bottom path for a total cost of
$C(\s^*) = kn$. The performance ratio is $kn^{1/\beta}/(kn) = n^{1/\beta-1}$,
which can again be made arbitrarily large.

Thus, for all classes of functions that include constant delay functions, the
LPOA is $\Omega(C(\s^*)^{1/\beta-1})$.

\section{Upper Bound on LPOA} \label{sec:ub}
For a game $\mathcal{G}$, let $\s^*$ be the socially optimal strategy profile
and let ${\s^*}\pbet$ be the socially optimal strategy profile for
$\mathcal{G}\pbet$ (i.e. if the true costs were those in $\mathcal{G}\pbet$).
Furthermore, let $\poab$ be the price of anarchy for games with cost functions in
$D\pbet$.
\begin{thm} \label{thm:lpoa}
  \[
    \text{LPOA} = O\p{{\poab}^{1/\beta} C(\s^*)^{1/\beta-1}}
  \]
\end{thm}
\begin{proof}
  We assume that there exists some constant $\xi$ such that either
  $\Delta_e(x_e) \ge \xi$ or $\Delta_e(x_e) = 0$ for all $e \in E$
  \footnote{If this not the case, then Theorem~\ref{thm:lpoa} does not hold.
  There are pathological examples in which, as the limit as $\beta$ goes to 0,
  the LPOA is unbounded. However, for reasonable examples, we assume that this
  is not the case.}. If $\Delta_e(x_e) = 0$ or $\xi \ge 1$, then $c_e\pbet(x_e)
  = c_e(x_e)$. Otherwise,
  \begin{align*}
    c_e\pbet(x_e) &= \tau_e + \Delta_e(x_e)^\beta \\
    &= \tau_e + \Delta_e(x_e) + (\Delta_e(x_e)^\beta - \Delta_e(x_e)) \\
    &= c_e(x_e) + \Delta_e(x_e)(\Delta_e(x_e)^{\beta-1} - 1) \\
    &\le c_e(x_e) + c_e(x_e)(\Delta_e(x_e)^{\beta-1} - 1) \\
    &= c_e(x_e) (1 + \Delta_e(x_e)^{\beta-1}-1) \\
    &= c_e(x_e) \Delta_e(x_e)^{\beta-1} \\
    &\le \xi^{\beta-1} c_e(x_e)
  \end{align*}
  In either case,
  \begin{align*}
    c_e\pbet(x_e) &\le \xi^{\beta-1} c_e(x_e) \quad \text{for all }e \in E
  \end{align*}
  Thus,
  \[
    C\pbet({\s^*}\pbet) \le C\pbet(\s^*) \le {\xi'}^{\beta-1} C(\s^*)
  \]
  where $\xi' = \min\{\xi,1\}$. By definition,
  \begin{align*}
  \frac{C\pbet(\s\pbet)}{C\pbet({\s^*}\pbet)} &\le \poab \\
  \Rightarrow \frac{C\pbet(\s\pbet)}{\poab} &\le C\pbet({\s^*}\pbet) \\
  \Rightarrow \frac{C\pbet(\s\pbet)}{\poab} &\le {\xi'}^{\beta-1} C(\s^*)
  \end{align*}
  Next, we observe that
  \begin{align*}
    C\pbet(\s\pbet)^{1/\beta} &= \p{\sum_{e \in E} x_e\p{\tau_e +
    \Delta_e(x_e)^\beta}}^{1/\beta} \\
    &\ge \sum_{e \in E} x_e^{1/\beta}\p{\tau_e^{1/\beta} + \Delta_e(x_e)} \\
    &\ge \sum_{e \in E} x_e\p{\tau_e + \Delta_e(x_e)} \\
    &= C(\s\pbet)
  \end{align*}
  Putting both parts together, we have
  \begin{align*}
    \frac{C(\s\pbet)}{{\poab}^{1/\beta}} &\le {\xi'}^{1-1/\beta}
    C(\s^*)^{1/\beta} \\
    \frac{C(\s\pbet)}{C(\s^*)} &\le {\poab}^{1/\beta} {\xi'}^{1-1/\beta}
    C(\s^*)^{1/\beta-1} \\
    &= O\p{{\poab}^{1/\beta} C(\s^*)^{1/\beta-1}}
  \end{align*}
\end{proof}


\section{Finiteness of POA$\pbet$} \label{sec:ubl}
We begin by defining the notion of \textit{smoothness}, as proposed by
Roughgarden \cite{Roughgarden2012}.
\begin{defn}
  A cost function $c$ is $(\lambda,\mu)$-smooth for $\lambda \ge 0$, $\mu < 1$
  if for any $x,x' \in \Z_+$, it holds that
  \begin{equation}
    c(x+1) x' \le \lambda x' c(x') + \mu x c(x).
    \label{eq:lms1}
  \end{equation}
  Furthermore, a family of cost functions $D$ is $(\lambda,\mu)$-smooth if every
  $c \in D$ is $(\lambda,\mu)$-smooth
\end{defn}A game $\mathcal{G}$ is $(\lambda,\mu)$-smooth if $D$ is $(\lambda,\mu)$-smooth.
For a family $\hat{\mathcal{G}}$ of $(\lambda,\mu)$-smooth games, the price of
anarchy for that game is at most $\frac{\lambda}{1-\mu}$ \cite{Roughgarden2012}.
Moreover, if $\hat{\mathcal{G}}$ has a price of anarchy, then there exists
some $(\lambda,\mu)$ such that $\hat{\mathcal{G}}$ is $(\lambda,\mu)$-smooth and
$\frac{\lambda}{1-\mu} = \poa$ \cite{Roughgarden2012}. Using this, we will show
that if POA is constant for $\hat{\mathcal{G}}$, then \poab is constant as well.

\begin{thm} \label{thm:smooth}
  If $D$ is $(\lambda,\mu)$-smooth, then $D\pbet$ is $(\max\{1,\lambda^\beta +
  \mu^\beta\},\mu^\beta)$-smooth.
\end{thm}
\begin{proof}
  Consider any $c \in D$. Since $\Delta(x) = c(x) - \tau$ belongs to the same
  class of functions as $c(x)$, $\Delta(x)$ must also be $(\lambda,\mu)$-smooth.
  We will show that $\Delta\pbet(x) = (c(x) - \tau)^\beta$ is $(\lambda^\beta +
  \mu^\beta,\mu^\beta)$-smooth.

  By definition of smoothness, we know that $\Delta(x+1) x' \le \lambda x'
  \Delta(x') + \mu x \Delta(x)$. Raising both sides to the power $\beta$, we
  have
  \begin{align*}
    (\Delta(x+1)x')^\beta &\le \p{\lambda x'\Delta(x') + \mu x\Delta(x)}^\beta \\
    \Delta(x+1)^\beta {x'}^\beta &\le \lambda^\beta {x'}^\beta \Delta(x')^\beta
    + \mu^\beta x^\beta \Delta(x)^\beta \\
    \Delta(x+1)^\beta x' &\le \lambda^\beta x' \Delta(x')^\beta + \mu^\beta
    x^\beta {x'}^{1-\beta} \Delta(x)^\beta \\
    \Delta\pbet(x+1) x' &\le \lambda^\beta x' \Delta\pbet(x') + \mu^\beta
    x^\beta {x'}^{1-\beta} \Delta\pbet(x) \numberthis \label{eq:smooth}
  \end{align*}
  We consider two cases: either $x \ge x'$ or
  $x < x'$. If $x \ge x'$, then~\eqref{eq:smooth} becomes
  \[
    \Delta\pbet(x+1) x' \le \lambda^\beta x' \Delta\pbet(x') + \mu^\beta x
    \Delta\pbet(x).
  \]
  If $x < x'$, then $\Delta\pbet(x+1) \le \Delta\pbet(x')$, so
  \[
    \Delta\pbet(x+1) x' \le (\lambda^\beta + \mu^\beta) x' \Delta\pbet(x').
  \]
  In either case,
  \begin{equation}
    \Delta\pbet(x+1) x' \le (\lambda^\beta + \mu^\beta) x' \Delta\pbet(x') +
    \mu^\beta x \Delta\pbet(x).
    \label{eq:deltasmooth}
  \end{equation}
  Finally, we must show that the smoothness equation holds for $c\pbet$. To do
  so, we observe that there are again two cases: Either $\lambda^\beta +
  \mu^\beta \ge 1$ or not. If $\lambda^\beta + \mu^\beta \ge 1$, then
  \[
    \tau x' \le \tau(\lambda^\beta + \mu^\beta)x' + \tau\mu^\beta x,
  \]
  meaning
  \begin{align*}
    (\Delta\pbet(x+1) + \tau) x' &\le (\lambda^\beta + \mu^\beta) x'
    (\Delta\pbet(x') + \tau) \\
    &+ \mu^\beta x (\Delta\pbet(x) + \tau) \\
    c\pbet(x+1) x' &\le (\lambda^\beta + \mu^\beta) x' c\pbet(x') +
    \mu^\beta x c\pbet(x).
  \end{align*}
  Otherwise, $\lambda^\beta + \mu^\beta < 1$, so~\eqref{eq:deltasmooth} becomes
  \[
    \Delta\pbet(x+1) x' \le x' \Delta\pbet(x') +
    \mu^\beta x \Delta\pbet(x).
  \]
  Furthermore,
  \[
    \tau x' \le \tau x' + \tau\mu^\beta x,
  \]
  so
  \begin{align*}
    (\Delta\pbet(x+1) + \tau) x' &\le x' (\Delta\pbet(x') + \tau)
    + \mu^\beta x (\Delta\pbet(x) + \tau) \\
    c\pbet(x+1) x' &\le (\lambda^\beta + \mu^\beta) x' c\pbet(x') +
    \mu^\beta x c\pbet(x).
  \end{align*}
  Since this holds for all $c\pbet \in D\pbet$, $D\pbet$ is
  $(\max\{\lambda^\beta + \mu^\beta\}, \mu^\beta)$-smooth.
\end{proof}

From Roughgarden, we know that for any $D$ with finite $\poa$, there exists
$\lambda,\mu$ such that $D$ is $(\lambda,\mu)$-smooth \cite{Roughgarden2012}.
Since $\poab \le \max\left\{\frac{\lambda^\beta +
\mu^\beta}{1-\mu^\beta}, \frac{1}{1-\mu^\beta}\right\}$, if $D$ has finite
$\poa$, then \poab is finite as well. This yields our main result:
\begin{thm} \label{thm:main}
 For any class of delay functions with finite $\poa$, $\lpoa =
 \Theta(C(\s^*)^{1/\beta-1})$.
\end{thm}
\begin{proof}
  The lower bound comes from Section~\ref{sec:lb}, while the upper bound comes
  from combining Theorems~\ref{thm:lpoa} and~\ref{thm:smooth}.
\end{proof}

\subsection{An Explicit Bound on \poab for Polynomial Delay Functions}
If $D$ contains only polynomial delay functions of degree at most $d$, i.e.
$c_e(x) = \sum_{i=0}^d a_{e,i} x_e^i$,\footnote{We use the convention that $0^0
= 1$.} we can construct an explicit bound for \poab and thus for $\lpoa$. We
so using analagous reasoning to Awerbuch et al., who show that for polynomial
delay functions of degree $d$, the price of anarchy is $O(2^d d^{d+1})$
\cite{Awerbuch2005}.

First, we need the following Lemma, proven by Awerbuch et al.
\cite{Awerbuch1995}.
\begin{lem} \label{lem:ln}
  For any $\kappa > 1$, $(x+y)^d \le \kappa x^d + (y(\frac{d}{\ln \kappa} +
  1))^d$.
\end{lem}
\noindent With this, we can prove the following theorem.
\begin{thm}
  For polynomial delay functions of degree at most $d$ and nonnegative
  coefficients $a_{e,i}$, $\poab = O(2^{d\beta} d^{d\beta+1})$.
\end{thm}
\begin{proof}
  %First, we observe that
  %\[
    %c_e\pbet(x_e) = \p{\sum_{i=0}^d a_{e,i} x_e^i}^\beta \le \sum_{i=0}^d
    %\p{a_{e,i} x_e^i}^\beta.
  %\]
  Let $\s$ be a Nash equilibrium for $\mathcal{G}\pbet$, and let $\s^*$ be the
  socially optimal strategy profile for $\mathcal{G}\pbet$. For each player $j$,
  the Nash equilibrium condition gives us
  \[
    \sum_{e \in s_j} c_e\pbet(x_e) \le \sum_{e \in s_j^*} c_e\pbet(x_e + 1)
  \]
  Summing over all players, we get
  \[
    C\pbet(\s) = \sum_{j=1}^k \sum_{e \in s_j} c_e\pbet(x_e) \le \sum_{e
    \in E} x_e^* c_e\pbet(x_e + 1)
  \]
  We can bound this by
  \begin{align*}
    \sum_{e \in E} x_e^* c_e\pbet(x_e + 1) &= \sum_{e \in E} x_e^*
    \p{\p{\sum_{i=0}^d a_{e,i} (x_e+1)^i}^\beta + \tau_e} \\
    &= \sum_{e \in E} x_e^* \tau_e + \sum_{e \in E} x_e^*
    \p{\sum_{i=0}^d a_{e,i} (x_e+1)^i}^\beta
  \end{align*}
  Observe that
  \begin{equation}
    \sum_{e \in E} x_e^* \tau_e \le C(\s^*) 
    \label{eq:tau}
  \end{equation}
  because $\tau_e \le C(x_e^*)$ for all $e$. The remaining term is
  \begin{align*}
    \sum_{e \in E} x_e^*& \p{\sum_{i=0}^d a_{e,i} (x_e+1)^i}^\beta \\
    &\le \sum_{e
    \in E} x_e^* \p{\sum_{i=0}^d a_{e,i} \p{\kappa x_e^i + \p{\frac{i}{\ln
    \kappa} + 1}^i}}^\beta \\
    &\le \sum_{e \in E} x_e^* \p{\sum_{i=0}^d a_{e,i} \p{\kappa x_e^i +
      \p{\frac{d}{\ln \kappa} + 1}^d}}^\beta \\
    &\le \kappa \sum_{e \in E} \sum_{i=0}^d a_{e,i}^\beta x_e^* x_e^{i\beta} \\
    &+ \p{\frac{d}{\ln \kappa} + 1}^{d\beta} \sum_{e \in E} x_e^*
    \p{\sum_{i=0}^d a_{e,i}}^\beta \numberthis \label{eq:twopart}
  \end{align*}
  where the first inequality comes from Lemma~\ref{lem:ln}, the second inequality
  follows because $(\frac{x}{\ln \kappa} + 1)^x$ is an increasing function for
  $x \ge 0$, and the third inequality holds because $V(x) = x^\beta$ is concave.
  Furthermore,
  \[
    x_e^* \p{\sum_{i=0}^d a_{e,i}}^\beta \le x_e^* \p{\sum_{i=0}^d a_{e,i}
    {x_e^*}^i}^\beta
  \]
  because either $x_e^* = 0$, in which case both sides are 0, or $x_e^* \ge 1$,
  in which case ${x_e^*}^i \ge 1$. This gives us
  \begin{align*}
    \p{\frac{d}{\ln \kappa} + 1}^{d\beta} \sum_{e \in E} x_e^* {\sum_{i=0}^d
    a_{e,i}}^\beta &\le \p{\frac{d}{\ln \kappa} + 1}^{d\beta} \sum_{e \in E}
    x_e^* \p{\sum_{i=0}^d a_{e,i}
    {x_e^*}^i}^\beta \\
    &= \p{\frac{d}{\ln \kappa} + 1}^{d\beta} C\pbet(\s^*).
  \end{align*}
  Next, we bound the other term in~\eqref{eq:twopart}, we can rearrange as
  \[
    \kappa\sum_{i=0}^d \sum_{e \in E} a_{e,i}^\beta x_e^* x_e^{i\beta}.
  \]
  We use Holder's Inequality, which states that for $\delta + \gamma = 1$,
  \[
    \sum_e v_e^\delta w_e^\gamma \le \p{\sum_e v_e}^\delta \p{\sum_e w_e}^\gamma.
  \]
  Choosing $\delta = \frac{i\beta}{i\beta + 1}$, $\gamma = \frac{1}{i\beta +
  1}$, $v_e = a_{e,i}^\beta x_e^{i\beta + 1}$, and $w_e = a_{e,i}^\beta
  {x_e^*}^{i\beta + 1}$, we have
  \begin{align*}
    \kappa\sum_{i=0}^d \sum_{e \in E} a_{e,i}^\beta x_e^* x_e^{i\beta} &\le
    \kappa \sum_{i=0}^d \p{\sum_{e \in E} a_{e,i}^\beta x_e^{i\beta +
    1}}^{i\beta/(i\beta+1)} \\
    &\cdot \p{\sum_{e \in E} a_{e,i}^\beta {x_e^*}^{i\beta +
    1}}^{1/(i\beta+1)}
  \end{align*}
  Since $a_{e,i}$ is nonnegative, for any $i$,
  \[
    a_{e,i}^\beta x_e^{i\beta+1} \le x_e \p{\sum_{\ell=0}^d a_{e,\ell}
    x_e^{\ell}}^\beta,
  \]
  as the left hand side is just one term in the summation on the right hand
  side and $V(x) = x^\beta$ is an increasing function. Since $C\pbet(\s) =
  \sum_{e \in E} x_e \p{\sum_{\ell=0}^d a_{e,\ell} x_e^\ell}^\beta$, we have
  \begin{align*}
    \kappa\sum_{i=0}^d \sum_{e \in E} a_{e,i}^\beta x_e^* x_e^{i\beta} &\le
    \kappa \sum_{i=0}^d C\pbet(\s)^{i\beta/(i\beta+1)}
    C\pbet(\s^*)^{1/(i\beta+1)} \\
  \end{align*}
  For $x \ge y \ge 0$ and $1 \ge \alpha \ge \alpha' \ge 0$,
  \[
    x^\alpha y^{1-\alpha} \ge x^{\alpha'} y^{1-\alpha'}.
  \]
  Using this,
  \begin{align*}
    \kappa\sum_{i=0}^d &C\pbet(\s)^{i\beta/(i\beta+1)}
    C\pbet(\s^*)^{1/(i\beta+1)} \\
    &\le \kappa\sum_{i=0}^d C\pbet(\s)^{d\beta/(d\beta+1)}
    C\pbet(\s^*)^{1/(d\beta+1)} \\
    &= \kappa(d+1)C\pbet(\s)^{d\beta/(d\beta+1)}
    C\pbet(\s^*)^{1/(d\beta+1)}
  \end{align*}
  Combining these results with~\eqref{eq:tau} and~\eqref{eq:twopart}, we
  have
  \begin{align*}
    C\pbet(\s) &\le \kappa(d+1)C\pbet(\s)^{d\beta/(d\beta+1)}
    C\pbet(\s^*)^{1/(d\beta+1)} \\
    &+ \p{\p{\frac{d}{\ln \kappa} + 1}^{d\beta} + 1} C\pbet(\s^*) \numberthis
    \label{eq:bigpoly}
  \end{align*}
  Let $\rho = (C\pbet(\s)/C\pbet(\s^*))^{1/d\beta+1}$. Dividing both sides
  of~\eqref{eq:bigpoly} by $C\pbet(\s^*)$, we get
  \[
    \rho^{d\beta+1} \le \kappa(d+1) \rho^{d\beta} + \p{\frac{d}{\ln \kappa} +
    1}^{d\beta} + 1
  \]
  Dividing both sides by $\rho^{d\beta}$, we get
  \[
    \rho \le \kappa(d+1) + \p{\frac{\frac{d}{\ln \kappa} +
    1}{\rho}}^{d\beta} + \frac{1}{\rho}.
  \]
  Choosing $\kappa = 2-\p{\frac{1}{2}}^{d\beta} - \frac{1}{2(d+1)}$, we get
  $\rho \le 2(d+1)$. This yields
  \begin{align*}
    \p{\frac{C\pbet(\s)}{C\pbet(\s^*)}}^{1/(d\beta+1)} &\le 2(d+1) \\
    \frac{C\pbet(\s)}{C\pbet(\s^*)} &\le \p{2(d+1)}^{d\beta+1} \\
    &= O(2^{d\beta} d^{d\beta+1}),
  \end{align*}
  which proves the theorem.
\end{proof}

%Furthermore, we take the definition of \textit{biased smoothness} from Meir and
%Parkes.
%\begin{defn}
  %The class of functions $D$ is $(\lambda,\mu)$-biased-smooth with respect to
  %$\beta$ if for any cost function $c \in D$ and any $x,x' \in \Z_+$,
  %\begin{equation}
    %c(x)x + c\pbet(x)(x' - x) \le \lambda c(x') x' + \mu c(x)x.
    %\label{eq:lmbs}
  %\end{equation}
%\end{defn}
%The following theorem, proved by Meir and Parkes for nonatomic congestion games,
%holds for atomic congestion games as well \cite{Meir2014}.
%\begin{thm}
  %For a game $\mathcal{G}$ with congestion functions $D$ $(\hat \lambda,\hat
  %\mu)$-biased-smooth, if $\s$ is an equilibrium and $\s^*$ is any other state,
  %then $C\pbet(\s) \le \frac{\hat \lambda}{1-\hat \mu} C\pbet(\s^*)$.
%\end{thm}
%\begin{proof}
  %By the Nash equilibrium condition, $\sum_{e \in E} c_e\pbet(x_e) x_e \le
  %\sum_{e \in E} c_e\pbet(x_e)x_e^*$. As noted by Meir and Parkes, this yields
  %\cite{Meir2014}
  %\begin{align*}
    %C\pbet(\s) &= \sum_{e \in E} c\pbet(x_e) x_e \\
    %&\le \sum_{e \in E} c_e\pbet(x_e) x_e^* \\
    %&\le \sum_{e \in E} c_e\pbet(x_e) x_e^* + c_e\pbet(x_e)(x_e^* - x_e) \\
    %&\le \sum_{e \in E} \hat \lambda c(x_e^*)x_e^* + \mu c\pbet(x_e)x_e
  %\end{align*}
%\end{proof}

\section{Discussion}
Throughout this paper, we have assumed that each agent has the same bias
parameter $\beta$. For a mixture of agents with different biases, one could ask
if an analogous upper bound on the LPOA can be derived.

Furthermore, while we have shown that $\poab$ is finite as long as $\poa$ is
finite, it is an open question whether $\poab \le \poa$ in all cases.  This could be provide more intuition as to how the bias places out in congestion games.

The present analysis of the lowball price of anarchy has two important results.  Firstly, the lowball price of anarchy grows with the size of the game or more specifically the cost of the optimal strategy.  And secondly, our approach suggests that congestion games are a good means of studying behavioral biases due to the nature and previous work done on congestion games.  In the future, we'd like to study other agents with behavioural biases in congestion games to better understand what happens in the worst and best cases.  Furthermore, we believe they should be studied as they help descriptively model the real world.

\printbibliography

% Any path $s$ that a player takes in $\mathcal{G}^\bet$ can be mapped to a path in $\mathcal{G}^\bet$

% We know from Rosenthal \cit that every atomic congestion game has a pure Nash equilibrium.  Thus in both $\mathcal{G}\pbet$ and 




\end{multicols}

\end{document}
