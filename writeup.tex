\documentclass[twoside]{article}
\usepackage{paper}

\title{\vspace{-15mm}\fontsize{24pt}{10pt}\selectfont\textbf{\titlee}} % Article title

\author{
\large
\textsc{Manish Raghavan, Serena Gupta}
\vspace{-5mm}
}

\begin{document}

\maketitle % Insert title

\thispagestyle{fancy} % All pages have headers and footers

\begin{abstract}

  Research shows that people aren't rational in evaluating costs --
  instead, they undervalue costs in a nonlinear fashion
  \cite{Kahneman1979}. Prospect theory seeks to model this
  behavior. We consider the effects of agents underestimating costs in atomic congestion
  games and bound the bias's negative effects on social performance. We extend the
  biased-smoothness framework proposed by Meir and Parkes to atomic congestion
  games and apply the framework to the cost-biased agents that prospect theory puts forth
  \cite{Meir2014}. We show regardless of the class of delay functions used, the
  ratio between the performance of biased agents and the social optimum grows
  with the size of the game. In other words, our findings suggest that people
  misunderstand the additivity of costs, preferring one large cost over a large
  number of smaller costs, even when it is not in their rational interest to do
  so.

\end{abstract}

\begin{multicols}{2} % Two-column layout throughout the main article text

\section{Introduction}
Prospect theory, introduced by Kahneman and Tversky in 1979, characterizes some
of the biases humans exhibit in decision-making \cite{Kahneman1979}. Roughly
speaking, it proposes the idea that humans tend to overweight small
probabilities and undervalue large rewards and costs. Experimental evidence
shows people are \textit{risk-seeking} when it comes to costs -- for example,
people prefer an 80\% chance of losing \$4,000 to a sure loss of \$3,000,
despite the expected loss being higher \cite{Kahneman1979}.

%This bias leads to suboptimal behavior

\section{A Motivating Example}

\section{Related Work}
The price of anarchy (POA) is defined as the worst-case ratio between a Nash
equilibrium and the socially optimal solution in a game, and intuitively gives
an upper bound on how closely the Nash equilibrium approximates the optimal
solution \cite{Koutsoupias2009}. It has been extensively studied in atomic
congestion games with rational agents, with tight matching lower and upper
bounds shown for various classes of cost funtions \cite{Aland2011}
\cite{Roughgarden2012}. Tim Roughgarden introduced the smooth analysis framework
to not only bound the price of anarchy, but also to construct worst-case
examples to provide matching lower bounds \cite{Roughgarden2012}.

While classical game theory relies on the assumption of rational agents, recent
work has sought to model agents with various behavioral biases. For example,
Reshef Meir and David Parkes consider the effects of behavioral biases in
nonatomic congestion games, applying smoothness arguments to upper bound the
price of anarchy \cite{Meir2014}.

\section{Overview of Results}
Broadly, we answer the question ``what is the price of anarchy when agents
underestimate costs?'' We call \textit{lowball agents}, agents who underestimate
costs following the behavioral models from prospect theory, and study the ratio between the true cost of the Nash equilibrium reached by lowball agents and the overall optimum solution in atomic congestion games.

In Section~\ref{sec:lb}, we show that for an atomic congestion game with
socially optimal strategy profile $\s^*$, the \textit{lowball price of anarchy} is
$\Omega(C(\s^*)^{1/\beta-1})$. In Sections~\ref{sec:ub} and~\ref{sec:ubl}, we
show a matching upper bound of $O(C(\s^*)^{1/\beta - 1})$.

\section{The Model}

Here we present a framework to model lowball agents in congestion games. We define a 
\textit{lowball agent} to evaluate costs according to the valuation function
\begin{equation}
  V(x) = x^{\beta}
  \label{eq:val}
\end{equation}
defined by Rieger and Wang for $0 < \beta \le 1$ \cite{Rieger2008}. For the
remainder of this paper, we assume that each agent has the same parameter
$\beta$.

An atomic congestion game, $\mathcal{G} = (G, D, P, k)$, is a
cost-minimization game parametrized by a directed graph $G$, a set of increasing
delay functions $D = \{c_e : \mathbb{Z}_{>0} \to \mathbb{R}\}$, a set of start
and end points for each player $P =\{(s_1, t_1), \cdots, (s_k, t_k)\}$, and $k$
players.

Lowball agents perceive congestion games by applying the valuation
function~\eqref{eq:val} to each edge's delay function, $c_e(x)$, leading them to
perceive the game as $\mathcal{G}\pbet= (G, D\pbet, P, k)$ where $D\pbet =
\{c_e\pbet(x)\}$ where $c_e\pbet(x) = (c_e(x))^\beta$. Within this new
congestion game $\mathcal{G}\pbet$, the lowball agents act rationally: however
the true costs they incur are reflected by the delay functions in $\mathcal{G}$.   

Let $S_i$ be the set of all strategies for player $i$ in $\mathcal{G}$ i.e. all
paths that go from $s_i$ to $t_i$.  Note $S_i$ is also the set of all strategies for
player $i$ in $\mathcal{G}\pbet$.  Since the true costs agents incur is
reflected in $\mathcal{G}$, it's useful to look at the difference in cost of an
agent $i$'s strategy $s_i \in S_i$, the path it takes in the graph, with the
costs functions in $\mathcal{G}\pbet$ versus the costs in $\mathcal{G}$.
Rational play in $\mathcal{G}\pbet$ amounts to the players playing according to
a Nash equilibrium.  Furthermore we know from Rosenthal \cite{Rosenthal1973}
that every atomic congestion game has a pure Nash equilibrium.  Thus for a given
pure Nash equilibrium strategy profile $\mathbf{s} = (s_1, \cdots, s_k)$ with
each $s_i \in S_i$ for each $i$, we can evaulate the perceived cost of
$\mathbf{s}$ (ie the cost in $\mathcal{G}\pbet$) and actual cost that will be
incurred (the cost in $\mathcal{G}$).

In addition, we'll define for a strategy profile $\mathbf{s}$, the true cost to
player $i$ to be: \[C_i(s) = \sum\limits_{e \in s_i}c_e(x_e)\] where
$\mathbf{x}$ is the vector of the number of players on each edge induced by
$\mathbf{s}$.  And the perceived cost to player $i$ is: \[C\pbet_i(s) =
\sum\limits_{e \in s_i}c_e\pbet(x_e) = \sum\limits_{e \in s_i}c_e(x_e)^\beta\]
From this, we know the total social cost is: \[C(\mathbf{s}) =
\sum_{i=1}^{k}C_i(\mathbf{s})\]
        
Traditionally we define the \textit{price of anarchy (POA)} to be the ratio
between the worst Nash equilibrium and the overall optimum solution
\cite{Koutsoupias2009}.  In more precise terms, the price of anarchy for a
family of games $\hat{\mathcal{G}}$ is \[\sup_{\mathcal{G} \in \hat{\mathcal{G}}}
\frac{C(\s)}{C(\s^*)}\] where $\s$ is a Nash equilibrium of $\mathcal{G}$ and
$\s^*$ is the overall optimum strategy profile of $\mathcal{G}$.

\begin{defn}
  The \textit{lowball price of anarchy (LPOA)} is
  \begin{equation}
    \sup_{\mathcal{G} \in \hat{\mathcal{G}}} \frac{C(\s\pbet)}{C(\s^*)}
    \label{eq:lpoa}
  \end{equation}
  where $\s\pbet$ is a Nash equilibrium of $\mathcal{G}\pbet$ and $\s^*$ is the
  socially optimum strategy profile of $\mathcal{G}$.
\end{defn}
In other words, the LPOA is a ratio between the true cost of the Nash
equilibrium reached by lowball agents and the overall optimum solution.

\section{Preliminaries}
\begin{lem} \label{lem:scale}
  Scaling the cost functions by a constant $a > 0$, i.e. $\hat c_e(x) = a
  c_e(x)$, does not change the ratio $C(\s\pbet)/C(\s^*)$.
\end{lem}
\begin{proof}
  $\s\pbet$ is a Nash equilibrium in $\mathcal{G}\pbet$ if and only if $\sum_{e
  \in s_i} c_e\pbet(x_e) \le \sum_{e \in s_i'} c_e\pbet(x_e + 1)$ for all $i$.
  After scaling the cost functions by $a$, $\s\pbet$ is still a Nash equilibrium
  because for all $i$, $\sum_{e \in s_i} \hat c_e\pbet(x_e) = a^\beta \sum_{e
  \in s_i} c_e\pbet(x_e) \le a^\beta \sum_{e \in s_i'} c_e\pbet(x_e + 1) =
  \sum_{e \in s_i'} \hat c_e\pbet(x_e+1)$.

  Since the strategy profile remains the same, the ratio between costs is $a
  C(\s\pbet)/(a C(\s^*)) = C(\s\pbet)/C(\s^*)$.
\end{proof}

Because $V(x) = x^\beta$ is concave, $V$ is subadditive, i.e.
\begin{equation}
  (x+y)^\beta \le x^\beta + y^\beta
  \label{eq:subadditive}
\end{equation}
for $x,y \ge 0$.

\section{Lower Bound on LPOA} \label{sec:lb}
\begin{figure}[H]
  \centering
  \begin{subfigure}[b]{\linewidth}
    \centering
    \begin{tikzpicture} [baseline=(s.base), arc/.style={->,thick,>=stealth},
      vertex/.style={draw,circle,minimum size=.5cm},scale=1.5]

      \foreach \l/\n/\p in {{s/s/(0,0)}, {v1/ /(1,0)}, {v2/ /(3,0)},
      {t/t/(4,0)}}
      {
        \node [vertex] at \p (\l) {$\n$};
      }
      \node at (2,0) (d) {$\dots$};
      \draw
      [-,thick,decorate,decoration={brace,amplitude=10pt}](3,-0.2) -- (0,-0.2)
      node[black,midway,yshift=-0.5cm]
      {$n$};

      \foreach \a/\b in {{s/v1}, {v1/d}, {d/v2}, {v2/t}}
      {
        \Arc {\a}{\b}{1}
      }

      \draw [arc,bend left=40] (s) to node [auto] {$n^{1/\beta}$} (t);
    \end{tikzpicture}
    \caption{A game $\mathcal{G}$ with non-constant LPOA}
    \label{fig:lower}
  \end{subfigure}

  \begin{subfigure}[b]{\linewidth}
    \centering
    \begin{tikzpicture} [baseline=(s.base), arc/.style={->,thick,>=stealth},
      vertex/.style={draw,circle,minimum size=.5cm},scale=1.5]

      \foreach \l/\n/\p in {{s/s/(0,0)}, {v1/ /(1,0)}, {v2/ /(3,0)},
      {t/t/(4,0)}}
      {
        \node [vertex] at \p (\l) {$\n$};
      }
      \node at (2,0) (d) {$\dots$};
      \draw
      [-,thick,decorate,decoration={brace,amplitude=10pt}](3,-0.2) -- (0,-0.2)
      node[black,midway,yshift=-0.5cm]
      {$n$};

      \foreach \a/\b in {{s/v1}, {v1/d}, {d/v2}, {v2/t}}
      {
        \Arc {\a}{\b}{1}
      }

      \draw [arc,bend left=40] (s) to node [auto] {$n$} (t);
    \end{tikzpicture}
    \caption{The game $\mathcal{G}\pbet$ that the agent perceives}
    \label{fig:lowerbet}
  \end{subfigure}
  \caption{A lower bound on the LPOA}
\end{figure}

As a lower bound, consider the LPOA in the one-player game shown in
Figure~\ref{fig:lower}. Through its valuation function~\eqref{eq:val}, the agent
perceives the game as in Figure~\ref{fig:lowerbet}. Thus, it is indifferent
between the upper and lower paths, so taking the upper path is a valid
equilibrium. However, the overall cost of this strategy is $n^{1/\beta}$,
whereas the optimal social cost is $C(\s^*) = n$, corresponding to the agent
taking the lower path. Thus, the performance ratio is $n^{1/\beta}/n =
n^{1/\beta-1} = C(\s^*)^{1/\beta-1}$.

Considering the cost functions $c_e$ to be constants, this can be extended to
arbitrarily many players: if $k$ players play this game, a valid Nash
equilibrium $\s\pbet$ corresponds to all players taking the top path, incurring
a total cost of $C(\s\pbet) = kn^{1/\beta}$, when the optimal strategy profile
$\s^*$ corresponds to all agents taking the bottom path for a total cost of
$C(\s^*) = kn$. The performance ratio is $kn^{1/\beta}/(kn) = n^{1/\beta-1}$,
which can again be made arbitrarily large.

Thus, for all classes of functions that include constant delay functions, the
LPOA is $\Omega(C(\s^*)^{1/\beta-1})$.

\section{Upper Bound on LPOA} \label{sec:ub}
For a game $\mathcal{G}$, let $\s^*$ be the socially optimal strategy profile
and let ${\s^*}\pbet$ be the socially optimal strategy profile for
$\mathcal{G}\pbet$ (i.e. if the true costs were those in $\mathcal{G}\pbet$).
\begin{thm} \label{thm:lpoa}
  \[
    \text{LPOA} \le {\poab}^{1/\beta} C(\s^*)^{1/\beta-1}
  \]
\end{thm}
\begin{proof}
  Since by Lemma~\ref{lem:scale} scaling $\mathcal{G}$ doesn't change the ratio
  between $C(\s\pbet)$ and $C(\s^*)$, we can assume either $c_e(x_e) = 0$ or
  $c_e(x_e) \ge 1$ for all $e \in E$. Thus, $c_e\pbet(x_e) \le c_e(x_e)$. This
  means that $C\pbet({\s^*}\pbet) \le C\pbet(\s^*) \le C(\s^*)$.

  By definition, we have $\poab \ge C\pbet(\s\pbet)/C\pbet({\s^*}\pbet)$, so
  $C\pbet({\s^*}\pbet) \ge C\pbet(\s\pbet)/\poab$. Next, we observe that
  \begin{align*}
    C\pbet(\s\pbet)^{1/\beta} &= \p{\sum_{e \in E} x_e c(x_e)^\beta}^{1/\beta}
    \\
    &\ge \sum_{e \in E} x_e^{1/\beta} c(x_e) \\
    &\ge \sum_{e \in E} x_e c(x_e) \\
    &= C(\s\pbet)
  \end{align*}
  Putting this all together, we have
  \begin{align*}
    \frac{C(\s\pbet)}{{\poab}^{1/\beta}} &\le C(\s^*)^{1/\beta} \\
    \frac{C(\s\pbet)}{C(\s^*)} &\le {\poab}^{1/\beta} C(\s^*)^{1/\beta-1}
  \end{align*}
\end{proof}


\section{Upper Bound on POA$\pbet$} \label{sec:ubl}
We begin by defining the notion of \textit{smoothness}, as proposed by
Roughgarden \cite{Roughgarden2012}.
\begin{defn}
  A cost function $c$ is $(\lambda,\mu)$-smooth for $\lambda \ge 0$, $\mu < 1$
  if for any $x,x' \in \Z_+$, it holds that
  \begin{equation}
    c(x+1) x' \le \lambda x' c(x') + \mu x c(x).
    \label{eq:lms1}
  \end{equation}
  Furthermore, a family of cost functions $D$ is $(\lambda,\mu)$-smooth if every
  $c \in D$ is $(\lambda,\mu)$-smooth
\end{defn}
A game $\mathcal{G}$ is $(\lambda,\mu)$-smooth if $D$ is $(\lambda,\mu)$ smooth.
For a family $\hat{\mathcal{G}}$ of $(\lambda,\mu)$-smooth games, the price of
anarchy for that game is at most $\frac{\lambda}{1-\mu}$ \cite{Roughgarden2012}.
Moreover, if $\hat{\mathcal{G}}$ has a price of anarchy POA, then there exists
some $(\lambda,\mu)$ such that $\hat{\mathcal{G}}$ is $(\lambda,\mu)$-smooth and
$\frac{\lambda}{1-\mu} = \poa$ \cite{Roughgarden2012}. Using this, we will show
that if POA is constant for $\hat{\mathcal{G}}$, then $\poab$ is constant as
well.

\begin{thm} \label{thm:smooth}
  If $D$ is $(\lambda,\mu)$-smooth, then $D\pbet$ is
  $(\lambda^\beta + \mu^\beta,\mu^\beta)$-smooth.
\end{thm}
\begin{proof}
  By definition of smoothness, we know that for any $c \in D$, $c(x+1) x' \le
  \lambda x' c(x') + \mu x c(x)$. Raising both sides to the power $\beta$, we
  have
  \begin{align*}
    (c(x+1)x')^\beta &\le \p{\lambda x'c(x') + \mu xc(x)}^\beta \\
    c(x+1)^\beta {x'}^\beta &\le \lambda^\beta {x'}^\beta c(x')^\beta + \mu^\beta
    x^\beta c(x)^\beta \\
    c(x+1)^\beta x' &\le \lambda^\beta x' c(x')^\beta + \mu^\beta x^\beta
    {x'}^{1-\beta} c(x)^\beta \\
    c\pbet(x+1) x' &\le \lambda^\beta x' c\pbet(x') + \mu^\beta x^\beta
    {x'}^{1-\beta} c\pbet(x) \numberthis \label{eq:smooth}
  \end{align*}
  We consider two cases: either $x \ge x'$ or
  $x < x'$. If $x \ge x'$, then~\eqref{eq:smooth} becomes
  \[
    c\pbet(x+1) x' \le \lambda^\beta x' c\pbet(x') + \mu^\beta x
    c\pbet(x).
  \]
  If $x < x'$, then $c\pbet(x+1) \le c\pbet(x')$, so
  \[
    c\pbet(x+1) x' \le (\lambda^\beta + \mu^\beta) x' c\pbet(x').
  \]
  In either case,
  \[
    c\pbet(x+1) x' \le (\lambda^\beta + \mu^\beta) x' c\pbet(x') + \mu^\beta x
    c\pbet(x).
  \]
  Since this holds for all $c\pbet \in D\pbet$, $D\pbet$ is $(\lambda^\beta +
  \mu^\beta, \mu^\beta)$ smooth.
\end{proof}

From Roughgarden, we know that for any $D$ with finite $\poa$, there exists
$\lambda,\mu$ such that $D$ is $(\lambda,\mu)$-smooth \cite{Roughgarden2012}.
Since $\poab \le \frac{\lambda^\beta + \mu^\beta}{1-\mu^\beta}$, if $D$ has
finite $\poa$, then $\poab$ is finite as well. This yields our main result:
\begin{thm} \label{thm:main}
 For any class of delay functions with finite $\poa$, $\lpoa =
 \Theta(C(\s^*)^{1/\beta-1})$.
\end{thm}
\begin{proof}
  The lower bound comes from Section~\ref{sec:lb}, while the upper bound comes
  from combining Theorems~\ref{thm:lpoa} and~\ref{thm:smooth}.
\end{proof}


%Furthermore, we take the definition of \textit{biased smoothness} from Meir and
%Parkes.
%\begin{defn}
  %The class of functions $D$ is $(\lambda,\mu)$-biased-smooth with respect to
  %$\beta$ if for any cost function $c \in D$ and any $x,x' \in \Z_+$,
  %\begin{equation}
    %c(x)x + c\pbet(x)(x' - x) \le \lambda c(x') x' + \mu c(x)x.
    %\label{eq:lmbs}
  %\end{equation}
%\end{defn}
%The following theorem, proved by Meir and Parkes for nonatomic congestion games,
%holds for atomic congestion games as well \cite{Meir2014}.
%\begin{thm}
  %For a game $\mathcal{G}$ with congestion functions $D$ $(\hat \lambda,\hat
  %\mu)$-biased-smooth, if $\s$ is an equilibrium and $\s^*$ is any other state,
  %then $C\pbet(\s) \le \frac{\hat \lambda}{1-\hat \mu} C\pbet(\s^*)$.
%\end{thm}
%\begin{proof}
  %By the Nash equilibrium condition, $\sum_{e \in E} c_e\pbet(x_e) x_e \le
  %\sum_{e \in E} c_e\pbet(x_e)x_e^*$. As noted by Meir and Parkes, this yields
  %\cite{Meir2014}
  %\begin{align*}
    %C\pbet(\s) &= \sum_{e \in E} c\pbet(x_e) x_e \\
    %&\le \sum_{e \in E} c_e\pbet(x_e) x_e^* \\
    %&\le \sum_{e \in E} c_e\pbet(x_e) x_e^* + c_e\pbet(x_e)(x_e^* - x_e) \\
    %&\le \sum_{e \in E} \hat \lambda c(x_e^*)x_e^* + \mu c\pbet(x_e)x_e
  %\end{align*}
%\end{proof}

\section{Open Questions}

\printbibliography

% Any path $s$ that a player takes in $\mathcal{G}^\bet$ can be mapped to a path in $\mathcal{G}^\bet$

% We know from Rosenthal \cit that every atomic congestion game has a pure Nash equilibrium.  Thus in both $\mathcal{G}\pbet$ and 




\end{multicols}

\end{document}
