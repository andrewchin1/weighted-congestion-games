\documentclass[twoside]{article}
\usepackage[sc]{mathpazo} % Use the Palatino font
\usepackage[T1]{fontenc} % Use 8-bit encoding that has 256 glyphs
\linespread{1.05} % Line spacing - Palatino needs more space between lines
\usepackage{microtype} % Slightly tweak font spacing for aesthetics
\usepackage[hmarginratio=1:1,left=17mm, bottom=17mm, top=32mm,columnsep=15pt]{geometry} % Document margins
\usepackage{multicol} % Used for the two-column layout of the document
\usepackage[hang, small,labelfont=bf,up,textfont=it,up]{caption} % Custom captions under/above floats in tables or figures
\usepackage{booktabs} % Horizontal rules in tables
\usepackage{float} % Required for tables and figures in the multi-column environment - they need to be placed in specific locations with the [H] (e.g. \begin{table}[H])
\usepackage{hyperref} % For hyperlinks in the PDF

\usepackage{abstract} % Allows abstract customization
\renewcommand{\abstractnamefont}{\normalfont\bfseries} % Set the "Abstract" text to bold
\renewcommand{\abstracttextfont}{\normalfont\small\itshape} % Set the abstract itself to small italic text

\usepackage{titlesec} % Allows customization of titles
\renewcommand\thesection{\Roman{section}} % Roman numerals for the sections
\renewcommand\thesubsection{\Roman{subsection}} % Roman numerals for subsections
\titleformat{\section}[block]{\large\scshape\centering}{\thesection.}{1em}{} % Change the look of the section titles
\titleformat{\subsection}[block]{\large}{\thesubsection.}{1em}{} % Change the look of the section titles
\usepackage[backend=biber,url=false]{biblatex}
\bibliography{congestion}

\usepackage{fancyhdr} % Headers and footers
\pagestyle{fancy} % All pages have headers and footers
\fancyhead{} % Blank out the default header
\fancyfoot{} % Blank out the default footer
\def\titlee{LowBALLIN'}
\fancyhead[C]{\titlee $\bullet$ Spring 2016 $\bullet$ Papadimitriou CS270} % Custom header text
\fancyfoot[RO,LE]{\thepage} % Custom footer text

\newcommand{\pbet}{^{(\beta)}}
\newcommand{\s}{\mathbf{s}}
\newcommand{\fix}{\textbf{FIX!!}}
\newcommand{\cit}{\textbf{CITE!!}}


%----------------------------------------------------------------------------------------
%	TITLE SECTION
%----------------------------------------------------------------------------------------

\title{\vspace{-15mm}\fontsize{24pt}{10pt}\selectfont\textbf{\titlee}} % Article title

\author{
\large
\textsc{Manish Raghavan, Serena Gupta}
\vspace{-5mm}
}

%----------------------------------------------------------------------------------------

\begin{document}

\maketitle % Insert title

\thispagestyle{fancy} % All pages have headers and footers

%----------------------------------------------------------------------------------------
%	ABSTRACT
%----------------------------------------------------------------------------------------

\begin{abstract}

\noindent % Add shit in

\end{abstract}

%----------------------------------------------------------------------------------------
%	ARTICLE CONTENTS
%----------------------------------------------------------------------------------------

\begin{multicols}{2} % Two-column layout throughout the main article text

\section{Introduction}

\section{A Motivating Example}

\section{Related Work}

\section{Overview of Results}

\section{Model Overview}

Here we present a framework to model \textit{lowball agents} in a congestion game.  An atomic congestion game, $\mathcal{G} = (G, D, P, k)$, is a cost-minimization game parametrized by a directed graph $G$, $D$ a set of increasing delay functions $\{c_e : \mathbb{Z}_{>0} \to \mathbb{R}\}$, a set of start and end points for each player $P =\{(s_1, t_1), \cdots, (s_k, t_k)\}$ and $k$ players.  Recall from before, a lowball agent has a valuation function that is $V(x) = x^{\beta}$ for $0 < \beta \le 1$.  
Lowball agents perceive congestion games by applying $V(x)$ to each edge's delay function, $c_e(x)$, and this leads to them perceiving the congestion as $\mathcal{G}\pbet= (G, D\pbet, P, k)$ where $D\pbet$ is the set of delay functions from $\{c_e\pbet(x) = (c_e(x))^\beta\}$ \fix.  Within this new congestion game $\mathcal{G}\pbet$, the lowball agents act rationally: however the true costs they incur are reflected by the delay functions in $\mathcal{G}$.   

Let $S_i$ be the set of all strategies for player $i$ in $\mathcal{G}$ i.e. all
paths go from $s_i$ to $t_i$.  Note $S_i$ is also the set of all strategies for
player $i$ in $\mathcal{G}\pbet$.  Since the true costs agents incur is
reflected in $\mathcal{G}$, it's useful to look at the difference in cost of an
agent $i$'s strategy $s_i \in S_i$, the path it takes in the graph, with the
costs functions in $\mathcal{G}\pbet$ versus the costs in $\mathcal{G}$.
Rational play in $\mathcal{G}\pbet$ amounts to the players playing according to
a Nash equililbrium.  Furthermore we know from Rosenthal \cite{Rosenthal1973} that every atomic congestion game has a pure Nash equilibrium.  Thus for a given pure Nash equilibrium strategy profile $\mathbf{s} = (s_1, \cdots, s_k)$ with each $s_i \in S_i$ for each $i$, we can evaulate the perceived cost of $\mathbf{s}$ (ie the cost in $\mathcal{G}\pbet$) and actual cost that will be incurred (the cost in $\mathcal{G}$).

In addition, we'll define for a strategy profile $\mathbf{s}$, the true cost to player $i$ to be: \[C_i(s) = \sum\limits_{e \in s_i}c_e(x_e)\] where $\mathbf{x}$ is the vector of the number of players on each edge induced by $\mathbf{s}$.  And the perceived cost to player $i$ is: \[C\pbet_i(s) = \sum\limits_{e \in s_i}c_e\pbet(x_e) = \sum\limits_{e \in s_i}c_e(x_e)^\beta\]  From this, we know the total social cost is: \[C(\mathbf{s}) = \sum_{i=1}^{k}C_i(\mathbf{s})\]
        
Traditionally we define the \textit{price of anarchy (POA)}) to be the ratio
between the worst Nash equilibrium and the overall optimum solution
\cite{Koutsoupias2009}.  In more precise terms, the price of anarchy for a family of games $\hat{\mathcal{G}}$: \[\sup_{\mathcal{G} \in \hat{\mathcal{G}}} \frac{C(\s)}{C(\s^*)}\] where $\s$ is a Nash equilibrium of $\mathcal{G}$ and $\s^*$ is the overall optimum strategy profile of $\mathcal{G}$.

We'll define the \textit{lowball price of anarchy (LPOA)}) to be: \[\sup_{\mathcal{G} \in \hat{\mathcal{G}}} \frac{C(\s\pbet)}{C(\s^*)}\] where $\s\pbet$ is a Nash equilibrium of $\mathcal{G}\pbet$ and $\s^*$ is the overall optimum strategy profile of $\mathcal{G}$.  In other words, the LPOA is a ratio between the true cost of the Nash equilibrium reached by lowball agents and the overall optimum solution.

\section{Lower Bound on LPOA}

\section{Upper Bound on LPOA}

\section{Upper Bound on POA$\pbet$}

\section{Open Questions}

\printbibliography

% Any path $s$ that a player takes in $\mathcal{G}^\bet$ can be mapped to a path in $\mathcal{G}^\bet$

% We know from Rosenthal \cit that every atomic congestion game has a pure Nash equilibrium.  Thus in both $\mathcal{G}\pbet$ and 




\end{multicols}

\end{document}
